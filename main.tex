\documentclass[12pt]{article}

  \usepackage[english]{babel}
  \usepackage{hyperref}
  \usepackage{fancyhdr}
  \usepackage[dvipsnames]{xcolor}
  \usepackage{listings}
  \usepackage{parcolumns}
  \usepackage{algorithm}
  \usepackage{algorithmicx}
  \usepackage{algpseudocode}
  \usepackage{enumitem}
  \usepackage{geometry}
  \usepackage{soul}
  \usepackage{graphicx}
  \usepackage{enumitem}
  \usepackage{csquotes}
  \usepackage{bookmark}
  \usepackage{mdframed}
  \usepackage{mathtools}
  \usepackage{amsmath}
  \usepackage{amsthm}
  \usepackage[toc]{appendix}
  \usepackage[
    backend=biber,%
    style=ieee%
  ]{biblatex}

  % Bibliography Setup
  \addbibresource{main.bib}
  \newcommand{\CiteSection}[2]{%
    (\autocite{#1}, ~\S {#1})
  }

  % Theorem Environments
  \theoremstyle{definition}
  \newtheorem*{defn*}{Definition}
  \theoremstyle{plain}
  \newtheorem*{equ*}{Equation}

  % Definitions for Algorithmic Environments
  \algdef{SE}[VARIABLES]{GVariables}{EndGVariables}
    {\algorithmicvariables}
    {\algorithmicend\ \algorithmicvariables}
  \algnewcommand{\algorithmicvariables}{\textbf{global variables}}

  \algdef{SE}[VARIABLES]{LVariables}{EndLVariables}
    {\algorithmiclvariables}
    {\algorithmicend\ \algorithmiclvariables}
  \algnewcommand{\algorithmiclvariables}{\textit{local variables}}

  \renewcommand{\algorithmicrequire}{\textbf{Input:}}
  \renewcommand{\algorithmicensure}{\textbf{Output:}}
  \renewcommand\thealgorithm{}

  % Settings for math-mode
  \makeatletter
  \def\mathcolor#1#{\@mathcolor{#1}}
  \def\@mathcolor#1#2#3{%
    \protect\leavevmode
    \begingroup
      \color#1{#2}#3%
    \endgroup
  }
  \makeatother


  % Image Directory
  \graphicspath{ {screenshots/} }
  % Hyperlink Setup
  \hypersetup{
    colorlinks = true,
    urlcolor = blue,
    linkcolor = blue,
    citecolor = blue
  }
  % Syntax-Highlight for Code Snippets
  \lstset{
    backgroundcolor=\color{white},
    breaklines=true,
    captionpos=b,
    frame=tb,
    tabsize=4,
    % numbers=left,
    showstringspaces=false,
    commentstyle=\color{Red},
    keywordstyle=\color{Aquamarine},
    stringstyle=\color{ForestGreen}
  }

  % Page and Text Layout
  \pagestyle{fancy}
  \geometry{%
    a4paper,%
    top=1in,%
    bottom=1in,%
    left=1in,%
    right=1in%
  }
  \setlength{\headheight}{15pt}

  \newenvironment{ldefinitions}
    {\left.\begin{aligned}}
    {\end{aligned}\right\rbrace}

  \title{Module 5 Concept Discussion}
  \author{Ashton Hellwig}
  \date{\today}
  \rhead{CSC160 Concept Discussion}

\begin{document}
  \maketitle
  \tableofcontents
  \lstlistoflistings
  \newpage


  \part{Initial Post}

    \section{Research Prompt}
      \begin{mdframed}
        Why does the compiler not find logic errors?
      \end{mdframed}

    \section{Response}
      The compiler checks the source code for syntax errors but not for logic
        errors. This is because logic errors are categorized as
        \texttt{run-time errors} and not \texttt{compile-time errors}
        \cite{pitts_2000}. Logic errors are only visible while
        \textit{running} the program \cite[Chapter 4]{malik_2015}. The simplest
        way to identify an error which only occurs at run-time is to use a
        debugging tool and place break points throughout the code until the issue
        is found.


  \newpage
  \part{Responses}

    \section{Response 1}
      \begin{mdframed}[backgroundcolor=green!20]
        Reply to \textbf{Jesse Andringa} (\textit{Post ID: 38639739})

        The reason there are so many different computer languages is because
          there are so many different applications of coding. Coding is used
          in most fields of work these days, so it must cover a vast amount
          of applications. Because there are so many different types of
          applications for coding, different languages are out there that
          work better than others. For example, Swift is a language that is
          commonly used when developing apps on the app store. The language
          was built to make it easy for app builders, but would not be
          convenient for other types of work possibly like computing for
          research, which VBA through excel does more conveniently.
          It is all about the convenience and efficiency of the language
          specific to the line of work it is doing. 
      \end{mdframed}
      I find it interesting that you would mention something like
        VBA (which stands for Visual Basic for Applications) as a
        common tool because this also goes to show another reason
        there are so many programming languages: some of them die.
        VBA is rarely used by anyone these days aside from the
        companies which invested far too much money into it when it
        was a popular scripting language for spreadsheets. Now,
        data scientists tend to use languages such as
        \href{https://www.r-project.org/}{R},
        \href{https://www.python.org/}{Python},
        \href{https://julialang.org/}{Julia} (my favorite),
        and to a lesser extent
        \href{https://www.mathworks.com/products/matlab.html}{MATLAB}. For
        example, the reason that the language Julia was created was in order to
        have a language that is as expressive and readable as Python
        but have the performance of a compiled-to-binary statically typed
        language such as C/Cpp. MATLAB has its own domain -- generally used
        by engineers in the hard sciences or signal processing with its
        plethora of tools and an intuitive IDE, python has been used in
        conjunction with Jupyter Notebooks in order to make data and statistical
        research easier to share and repeat.
        


    \section{Response 2}
      \begin{quote}
        Reply to \textbf{} (\textit{Post ID: })
      \end{quote}
      Placeholder

  % Bibliography
  \newpage
  \nocite{malik_2015}
  \printbibliography[
    heading=bibintoc,
    title={Bibliography}
  ]
\end{document}
