\documentclass[12pt]{article}

  \usepackage[english]{babel}
  \usepackage{hyperref}
  \usepackage{fancyhdr}
  \usepackage[dvipsnames]{xcolor}
  \usepackage{listings}
  \usepackage{parcolumns}
  \usepackage{algorithm}
  \usepackage{algorithmicx}
  \usepackage{algpseudocode}
  \usepackage{enumitem}
  \usepackage{geometry}
  \usepackage{soul}
  \usepackage{graphicx}
  \usepackage{enumitem}
  \usepackage{csquotes}
  \usepackage{bookmark}
  \usepackage{mdframed}
  \usepackage{mathtools}
  \usepackage{amsmath}
  \usepackage{amsthm}
  \usepackage[toc]{appendix}
  \usepackage[
    backend=biber,%
    style=ieee%
  ]{biblatex}

  % Bibliography Setup
  \addbibresource{main.bib}
  \newcommand{\CiteSection}[2]{%
    (\autocite{#1}, ~\S {#1})
  }

  % Theorem Environments
  \theoremstyle{definition}
  \newtheorem*{defn*}{Definition}
  \theoremstyle{plain}
  \newtheorem*{equ*}{Equation}

  % Definitions for Algorithmic Environments
  \algdef{SE}[VARIABLES]{GVariables}{EndGVariables}
    {\algorithmicvariables}
    {\algorithmicend\ \algorithmicvariables}
  \algnewcommand{\algorithmicvariables}{\textbf{global variables}}

  \algdef{SE}[VARIABLES]{LVariables}{EndLVariables}
    {\algorithmiclvariables}
    {\algorithmicend\ \algorithmiclvariables}
  \algnewcommand{\algorithmiclvariables}{\textit{local variables}}

  \renewcommand{\algorithmicrequire}{\textbf{Input:}}
  \renewcommand{\algorithmicensure}{\textbf{Output:}}
  \renewcommand\thealgorithm{}

  % Settings for math-mode
  \makeatletter
  \def\mathcolor#1#{\@mathcolor{#1}}
  \def\@mathcolor#1#2#3{%
    \protect\leavevmode
    \begingroup
      \color#1{#2}#3%
    \endgroup
  }
  \makeatother


  % Image Directory
  \graphicspath{ {screenshots/} }
  % Hyperlink Setup
  \hypersetup{
    colorlinks = true,
    urlcolor = blue,
    linkcolor = blue,
    citecolor = blue
  }
  % Syntax-Highlight for Code Snippets
  \lstset{
    backgroundcolor=\color{white},
    breaklines=true,
    captionpos=b,
    frame=tb,
    tabsize=4,
    % numbers=left,
    showstringspaces=false,
    commentstyle=\color{Red},
    keywordstyle=\color{Aquamarine},
    stringstyle=\color{ForestGreen}
  }

  % Page and Text Layout
  \pagestyle{fancy}
  \geometry{%
    a4paper,%
    top=1in,%
    bottom=1in,%
    left=1in,%
    right=1in%
  }
  \setlength{\headheight}{15pt}

  \newenvironment{ldefinitions}
    {\left.\begin{aligned}}
    {\end{aligned}\right\rbrace}

  \title{Module 5 Concept Discussion}
  \author{Ashton Hellwig}
  \date{\today}
  \rhead{CSC160 Concept Discussion}

\begin{document}
  \maketitle
  \tableofcontents
  \lstlistoflistings
  \newpage


  \part{Initial Post}

    \section{Research Prompt}
      \begin{mdframed}
        Why does the compiler not find logic errors?
      \end{mdframed}

    \section{Response}
      The compiler checks the source code for syntax errors but not for logic
        errors. This is because logic errors are categorized as
        \texttt{run-time errors} and not \texttt{compile-time errors}
        \cite{pitts_2000}. Logic errors are only visible while
        \textit{running} the program \cite[Chapter 4]{malik_2015}. The simplest
        way to identify an error which only occurs at run-time is to use a
        debugging tool and place break points throughout the code until the issue
        is found.


  \newpage
  \part{Responses}

    \section{Response 1}
      \begin{quote}
        Reply to \textbf{} (\textit{Post ID:})
      \end{quote}
      Placeholder

    \section{Response 2}
      \begin{quote}
        Reply to \textbf{} (\textit{Post ID: })
      \end{quote}
      Placeholder

  % Bibliography
  \newpage
  \nocite{malik_2015}
  \printbibliography[
    heading=bibintoc,
    title={Bibliography}
  ]
\end{document}
